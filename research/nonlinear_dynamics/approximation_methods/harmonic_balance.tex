\documentclass{article}
\usepackage{amsmath, amssymb}
\usepackage{graphicx}

\title{Harmonic Balance Method}
\author{Arun Periyal}
\date{2024}

\begin{document}

\maketitle

\tableofcontents

\section{Introduction}
The Harmonic Balance (HB) method is a powerful technique for analyzing nonlinear systems that exhibit periodic behavior. It approximates solutions using truncated Fourier series and converts the differential equations into algebraic equations.

\section{Mathematical Formulation}
Consider a general nonlinear system of the form:
\begin{equation}
    \dot{x} = f(x, t)
\end{equation}
where $x$ represents the state variables and $f(x,t)$ is a nonlinear function. The periodic solution is assumed to be of the form:
\begin{equation}
    x(t) = \sum_{n=0}^{N} a_n \cos(n\omega t) + b_n \sin(n\omega t)
\end{equation}
where $a_n$ and $b_n$ are Fourier coefficients, and $\omega$ is the fundamental frequency.

Substituting this approximation into the governing equation and matching the Fourier coefficients leads to a system of algebraic equations.

\section{Application to Duffing Oscillator}
The Duffing equation is given by:
\begin{equation}
    \ddot{x} + \delta \dot{x} + \alpha x + \beta x^3 = F \cos(\omega t)
\end{equation}
Using the first harmonic approximation:
\begin{equation}
    x(t) \approx A \cos(\omega t) + B \sin(\omega t)
\end{equation}
Substituting into the equation and equating Fourier coefficients gives algebraic equations for $A$ and $B$, which can be solved numerically.

\section{Conclusion}
The Harmonic Balance Method provides an efficient way to analyze nonlinear periodic systems and is widely used in engineering applications.

\end{document}

